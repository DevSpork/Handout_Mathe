\documentclass[12pt, letterpaper, twoside]{article}
\usepackage[utf8]{inputenc}
\usepackage{amssymb}
\usepackage{amsmath}

\title{Einführung in die Mathematik für Informatiker, Cheatsheet}
\author{Tobias Kadenbach}
\date{\today}
 
\begin{document}
\maketitle

\section{Satz von Euler Fermat}
Seien $ a,n \in  \mathbb{N} $ und ggt(a,n) = 1 dann gilt: \\
\begin{center}
$ a^{\phi (n)} \equiv 1 (mod n) $ \\
\end{center}
Rechenbeispiel: \\
Berechnen Sie die letzten zwei Ziffern der Zahl $ 211^{1043} $


\end{document}