\documentclass[12pt, letterpaper, twoside]{article}
\usepackage[utf8]{inputenc}
\usepackage{amssymb}
\usepackage{amsmath}
\usepackage[german]{babel}

\usepackage{hyperref}
\hypersetup{
    colorlinks,
    citecolor=black,
    filecolor=black,
    linkcolor=black,
    urlcolor=black
}

\newtheorem{theorem}{Satz}[section]
\newtheorem{proposition}{Proposition}[section]

\newcommand{\compl}[1]{\bar{#1}}
\newcommand{\elemof}[2]{$ #1 \in #2 $}
\newcommand{\forallin}[2]{$ \forall #1 \in #2 $}

\title{Einführung in die Mathematik für Informatiker, Cheatsheet}
\author{Tobias Kadenbach}
\date{\today}
 
\begin{document}
\maketitle

\tableofcontents 

\section{Gruppen}
\subsection{Definition}
$ (G, \circ) $ ist eine Gruppe falls:
\begin{itemize}
	\item abgeschlossen bzgl. $ \circ $
	\item assoziativ
	\item neutrales Element mit $ \exists e \in G $ $\forall g \in G e \circ g = g \circ e = g $
	\item Inverse: $ \forall g \in G $  $\exists g^{-1}  \in G $ $g \circ g^{-1} = g^{-1} \circ g = e $
\end{itemize}
\subsection{Darstellung und abelsche Gruppen}
\noindent
Eine möglische Darstellung einer Gruppe ist eine sogenannte Gruppentafel dabei wir jedes Element der Gruppe in eine Zeile und eine Spalte geschrieben und anschließen werden so die Elemente der Gruppe mit der Gruppenoperation verbunden. Das ausfüllen erfolgt dabei nach dem Sudokuprinzip (in jeder Zeile und Spalte darf jedes Element nur exakt einmal vorkommen).  
Eine Gruppe wird auch abelsche genannt falls diese zur Hauptdiagonale symmetrisch ist. \\

\subsection{Gruppenisomorphie}
\noindent
Eine Gruppe g ist Isomorph zu einer anderen Gruppe h wenn:
\begin{itemize}
	\item h ist selber eine Gruppe
	\item es exisitiert ein Isomorphismus der jedem Element der Gruppe g ein Element der Gruppe h eineindeutig zuordnet es muss gelten wenn $a \mapsto x$ , $b \mapsto y$ und $c \mapsto z$ und gilt $a \circ b = c$, dann muss auch $x \circ y = z$ gelten.
	\item die Homomorphie Eigenschaft ist erfüllt ( $ h(a \circ b) = h(a) \circ h(b) $ )
\end{itemize}
\subsection{Erzeugendensystem}
Ein Element einer Gruppe ist Erzeugensystem wenn mit diesem Element und der Gruppneoperation jedes Element der Gruppe erzeugt werden kann.
Beispiel: \\
($\mathbb{Z}_n , + $): $ m \in \mathbb{Z}_n $ ist Erzeuger gleichbedeutend sind:
\begin{itemize}
	\item m ist Einheit
	\item ggt(m,n) = 1
\end{itemize}

\noindent
( $ \mathbb{Z}_n $ ist \textbf{zyklisch}, denn z.B. $ <\{1\}>  =  \mathbb{Z}_n $ ) \\

\noindent
Die Anzahl an Erzeugern (o.a. Primitivwurzeln) lässt sich durch: $ \phi (n) $ errechnen.

Beispiel: \\
\indent
$ ( \mathbb{Z}_{13}, +)$ hat $ \phi(13) = 12 $ Erzeuger.



\subsection{Einheitengruppen}
Einheiten Gruppen $\mathbb{Z}_{n}^* $ enthalten nur die Einheiten der Gruppe $ \mathbb{Z}_n $

Beispiel: \\
Eine Primitvwurzel von $ \mathbb{Z}_{13}^* $ finden: \\
\begin{tabular}{c | c c c c c c c c c c c c c}
n & 1 & 2 & 3 & 4 & 5 & 6 & 7 & 8 & 9 & 10 & 11 & 12 \\
\hline
$ 2^n $& 2 & 4 & 8 & $ 16 \equiv 3 $ & 6 & 12 & $ 24 \equiv 11 $& $ 22 \equiv 9$ & $18 \equiv 5$ & 10 & $20 \equiv 7$ & $14 \equiv 1$ \\
\end{tabular}

\subsection{Untergruppen}
Sei $(G, \circ)$ eine Gruppe
$U \subset G$ ist eine Untergruppe von G falls:

\begin{itemize}
	\item U ist abgeschlossen bzgl.  $ \circ $
	\item $ e_G \in U $ (neutrales Element der Gruppe enthalten)
	\item $ \forall u \in U: u^{-1} \in U $ (für jedes Element auch inverses enthalten)
\end{itemize}

\begin{theorem}[Satz von Lagrange] Die Untergruppenordnung teilt die Gruppenordnung\end{theorem}

Beispiel:

$\mathbb{Z}_{14}^* $ kann Untergruppen der Ordnung 1,2,3,6 haben da $ |\mathbb{Z}_{14}^*| = 6 $
Dabei ist:
\begin{itemize}
	\item \{1\} Untergruppe der Ordnung  1
	\item \{1,13\} Untergruppe der Ordnung 2
	\item \{1,9,11\} Untergruppe der Ordnung 3
\end{itemize}

\subsection{Linksnebenklassen}

$U$ ist Linksnebenklasse von G falls $g \circ U = \{g\circ u | u \in U\}$ \\
\indent
Beispiel für $\mathbb{Z}_{14}^* $:\\
\indent
$1 \cdot U_2 = {1\cdot1,1\cdot13} = U_2 \cdot 13$\\
\indent
$3 \cdot U_2 = {3\cdot1,3\cdot13} = U_2 \cdot 11$\\
\indent
$5 \cdot U_2 = {5,9} = U_2 \cdot 9$\\

\subsection{Ordnung eines Elements}
$<g> = \{g, g \circ g, g\circ g \circ g\}$\\
$|<1>| = | \{1\} |= 1  \Rightarrow $ 1 hat die Ordnung 1

\section{Graphen}

\subsection{Komplemänterer Graph}
Formel um die Knotenanzahl von $ \compl{G} $ zu berechnen: $ \compl{G} = (V, \binom{V}{2} \ E )$ also $\binom{n}{2} - |E|$

%more todo

\subsection{Bipartite Grpahen}
Ein Grpah heißt bipartit falls man ihn mit genau zwei Farben so färben kann, dass nie zweimal die selbe Farbe nebeneinander auftritt.
Ein Graph kann nicht ipartit sein wenn es ungerade Kreise in dem Graphen gibt.

\subsection{Bäume}
Ein Baum ist ein Grpah der T=(V,E) der keine Kreise hat und zusammenhängend ist. 

\subsection{Subgraph}
Ein Graph H=(W,F) heißt Subgraph von G=(V,E), wenn $ W \subseteq V und F \subseteq E $.

\subsubsection{Induzierter Subgraph}
Ein induzierter Teilgraph entsteht, wenn beim Obergraphen ein oder mehrere Knoten samt dessen inzidenten \footnote{angrenzdende} Kanten entfernt werden. 

\subsection{Brücken}
Subgraph (\{u,v\}, \{\{u,v\}\}) von G, so dass für die Zusammenhangskomponente H von u und v gilt: (H, E(H)\{\{u,v\}\}) = s nicht zusammenhängend.

\subsection{Gelenkpunkte}
Wenn man diesen Knoten entfernt so ist der Grpah danach nicht mehr zusammenhängend.

\subsection{Blockgraph}
$ G_B =( V_B, E_B) $ \\
Knotenmenge $V_B = A \cup B$: \\
A: Menge der Gelenkpunkte \\
B: Menge der Blöcke

\subsection{k-facher Zusammenhang}
$ G=(V,E) $ ist k-fach zusammenhängend, falls:
	\begin{itemize}
		\item $|V|>K$
		\item $\forall \{x_1,...,x_{k-1}\} $
	\end{itemize}
	ist $G-\{x_1,...,x_{k-1}\}$ zusammenhängend.
	$\Leftrightarrow $ (Satz 61) für alle $a,b \in V$ gibt es k unabhängige Pfade zwischen a und b.

\begin{theorem}[Satz von 61]
für k=2: $G=(V,E) 2-fach$ zsh. $ \Leftrightarrow $ \forallin{a, b}{V}. \\
Es gibt 2 unabhängige Pfade von a nach b
 \end{theorem}
 \begin{proposition}
	 G 2-fach zusammenhängend $\Leftrightarrow$ G lässt sich aus einem Kreis durch sukzessives anhängen von Pfaden (Ohren) konstruieren.
 \end{proposition}

 \subsection{Eulerzüge}
 $G=(V,E)$ ist genau dann eulersch, wenn ein Weg bzw. Kreis $P$ existiert, für den gilt $G( |E| )=P( |E| )$
 Das bedeutet eulersche Graphen sind Graphen bei denen alle Kanten genau einmal durchlaufen werden müssen.

\section{Satz von Euler Fermat}
Seien $ a,n \in  \mathbb{N} $ und ggt(a,n) = 1 dann gilt: \\
\begin{center}
$ a^{\phi (n)} \equiv 1 (mod n) $ \\
\end{center}
Rechenbeispiel: \\
Berechnen Sie die letzten zwei Ziffern der Zahl $ 211^{1043} $:
Also 211 $ 211^{1043} $ mod 100 da 2 Stellen.



\end{document}